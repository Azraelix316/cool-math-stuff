\documentclass[12pt]{article}
\boldmath
\begin{document}
\begin{center}
    \section* {7.5: Graphing Quadratic Functions}
    \subsection* {Quadratic Forms} 
    ~\\
    \quad\quad Vertex: Used to find vertex of parabola, like completing the square but having the whole equation on one side (rebalance the equation).\\
    $y = a(x-h)^2 + c$ \\
    Where y is the dependent variable, x is the independent variable, a is a vertical scale, and h is a horizontal translation, and c is a vertical translation.
    (h,c) is the vertex of the parabola.
    The reason why this works is because (x-h) = 0, meaning that it is the maximum/minimum value at the point
    Axis of symmetry is h.
    You can transfer into vertex form by completing the square
    EX: $ y= x^2 - 4x +5$ \\
   $ y - 5 = x^2 - 4x = (x-2)^2$ \\
    $y=(x-2)^2 + 5$ \\~\\
    \qquad Standard: Just the typical \boldmath $ax^2+bx+c = y$ form, very useful to transition to other forms, otherwise it's pretty useless.\\
    Axis of symmetry is $-b/2a$ , which can be proved as the quadratic formula $x = -b+sqrt(b^2)$

\end{center}
\end{document}